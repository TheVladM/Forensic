\documentclass[14pt]{article}
\usepackage[utf8]{inputenc}
\usepackage[T1]{fontenc}
\usepackage[french]{babel}
\usepackage{geometry}
\usepackage{graphicx}
\usepackage{titlesec}
\usepackage{fancyhdr}
\usepackage{lastpage}
\usepackage{enumitem}
\usepackage{amsmath}
\usepackage{amssymb}
\usepackage{booktabs}
\usepackage{tabularx}
\usepackage{hyperref}
\hypersetup{
    colorlinks=true,
    linkcolor=blue,
    filecolor=magenta,      
    urlcolor=cyan,
}
\geometry{a4paper, margin=2.5cm}
\setlength{\parindent}{0pt}
\setlength{\parskip}{1em}
\titleformat{\section}{\Large\bfseries}{\thesection}{1em}{}
\titleformat{\subsection}{\large\bfseries}{\thesubsection}{1em}{}
\titleformat{\subsubsection}{\bfseries}{\thesubsubsection}{1em}{}

\begin{document}

\begin{titlepage}
    \centering
    \vspace*{1cm}
    {\Huge\bfseries École Nationale Supérieure Polytechnique de Yaoundé \par}
    \vspace{1.5cm}
    {\Huge\bfseries THEME : \\ Mener une recherche OSINT sur l'etudiante ONOMO NGONO ALICE BEATRICE \par}
    \vspace{1.5cm}
    {\Large Nom et Prenoms de l'etudiant: \textbf{MELONE} Andre Vladmir \par}
    \vspace{0.5cm}
    {\large Matricule: \textbf{22p059} \par}
    \vspace{0.5cm}
    {\large Unité d'Enseignement: \textbf{Introduction aux Techniques d'Investigation Numérique} \par}
    \vspace{0.5cm}
    {\large Classe:\textbf{ Niveau 4} \par}
    \vspace{0.5cm}
    {\large Filière: \textbf{Cybersécurité et Investigation Numérique} \par}
    \vspace{2cm}
    {\large Enseignant : \\ \textbf{Monsieur MINKA MI NGUIDJOI Thierry Emmanuel} \par}
    \vspace{1cm}

\end{titlepage}


\newpage
\tableofcontents
\newpage

\section{Ce que je sais deja avant de debuter :  }

Dans le cadre de cet exercice d’OSINT, la personne étudiée est une camarade que je connais depuis environ un an, nommée \textbf{\textit{ONOMO NGONO ALICE BEATRICE}}. Nous fréquentons le même établissement d’enseignement supérieur, ce qui m’a permis de recueillir certaines informations de base à son sujet.
Je connais ainsi son identité complète, sa filière et son niveau d’étude. Elle est actuellement inscrite en quatrième année du cycle ingénieur dans une spécialité liée à la \textbf{cybersécurité et à l’investigation numérique}, après avoir précédemment suivi une formation en ingénierie informatique dans un a l'\textbf{institut Africain d'Informatique (IAI Cameroun)}.
J’ai également accès à son contact téléphonique : \textbf{693055132}, obtenu dans un cadre académique . Ces éléments constituent le point de départ de la recherche, et serviront de base à la collecte d’informations issues de sources ouvertes, dans le respect des principes éthiques et de la vie privée.

\section{Ce que j'ai trouvé:}

Les recherches menées dans le cadre de cet exercice d’OSINT ont permis d’obtenir plusieurs informations pertinentes à partir de sources publiques, principalement issues de la plateforme Facebook. À partir du nom complet de la camarade, j’ai pu identifier un compte correspondant à son profil réel. L’analyse du contenu visible sur ce compte a permis de confirmer son identité grâce à la cohérence des informations affichées, notamment son parcours académique et certaines photos publiées. En explorant les interactions et la liste d’amis, il a également été possible d’identifier deux membres de sa famille : un frère et une cousine dont les profils sont eux aussi publics avec leurs noms reels et photos. Par ailleurs, j'ai pu retrouve sa ville d'origine a savoir \textbf{OKOLA}. Il est donc possible de deduire que son origine ethnique est \textbf{Eton}, tandis que la date de naissance de la cousine a pu être trouvée sur son propre profil. Toutes ces données proviennent exclusivement d’informations volontairement partagées sur les réseaux sociaux, accessibles sans autorisation particulière. Ces résultats démontrent la facilité avec laquelle il est possible de reconstituer des éléments personnels à partir de simples recherches ouvertes, soulignant l’importance de la prudence dans la gestion de son identité numérique.



\section{Recommandations:}

À la lumière des informations obtenues lors de cette recherche OSINT, plusieurs mesures peuvent être recommandées à la personne étudiée afin d’améliorer la protection de sa vie privée sur les réseaux sociaux, notamment Facebook :

\begin{enumerate}
    \item Renforcer les paramètres de confidentialité : vérifier régulièrement les paramètres de confidentialité du compte pour s’assurer que seules les personnes autorisées peuvent voir les publications, les informations de profil et la liste d’amis.

    \item Limiter la quantité d’informations personnelles partagées : éviter d’afficher publiquement des données sensibles comme la date de naissance, la ville d’origine, le numéro de téléphone ou les liens familiaux directs.

    \item Contrôler les publications et les identifications : examiner les anciennes publications et restreindre la possibilité d’être identifié ou mentionné par d’autres utilisateurs sans consentement.

    \item Utiliser des photos de profil neutres : privilégier des images qui ne permettent pas une identification trop précise du lieu de résidence, de l’établissement ou du cercle familial.

    \item Éviter les partages d’informations croisées : ne pas publier simultanément sur plusieurs plateformes les mêmes données personnelles, afin de limiter la corrélation d’informations entre différents comptes.

    \item Mettre en place une veille personnelle : rechercher régulièrement son propre nom sur les moteurs de recherche pour identifier d’éventuelles traces numériques non souhaitées.

    
\end{enumerate}

\vspace{1cm}



\end{document}